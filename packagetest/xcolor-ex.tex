\documentclass{article}
\usepackage[UTF8]{ctex}
\usepackage{tabularx}
\usepackage{longtable}
\usepackage{tabu}
\usepackage{amsmath}
\usepackage{multirow}
\usepackage{fancyhdr}
\usepackage[top=3cm, bottom=3cm, left=3cm, right=3cm]{geometry}
\usepackage{float}
\usepackage{booktabs}
\usepackage{lastpage}
\usepackage[dvipsnames,svgnames,table]{xcolor}
\usepackage{caption}% 解决表格标题与下方表格过近
\usepackage{csvsimple}
\usepackage[colorlinks=true]{hyperref}
\usepackage{cprotect}% http://ctan.org/pkg/cprotect
\usepackage{graphicx}
\usepackage{colortbl}
\usepackage{tikz}
%%%表格标题与前后文字间距%%%%%%%%%%%%%%%%%
\setlength{\abovecaptionskip}{2pt} 
\setlength{\belowcaptionskip}{2pt}
%%%%%%%%%%%%%%%%%%%%%%%%%%%%%%%%%%%%%%

\newcommand*{\mylogo}[1]{\begin{minipage}[c][2em][c]{2em}
 \includegraphics[width=1.5em]{#1}
  \end{minipage}
}

\pagestyle{fancy}
\fancyhead[C]{}
\fancyhead[L]{color/xcolor/colortbl宏包}
\fancyhead[R]{哈尔滨合正科技 \mylogo{logo.png}}
\fancyfoot[C]{\thepage\ / \pageref{LastPage}}
\renewcommand{\headrulewidth}{0.8pt}
\renewcommand{\footrulewidth}{0pt}
% 该命令来源于
% https://tex.stackexchange.com/questions/106984/how-to-draw-a-square-of-1cm-in-latex-filled-with-color
\newcommand{\mybox}[1]{\tikz{\path[draw=#1,fill=#1] (0,0) rectangle (1em,0.5em);}}

\title{color/xcolor/colortbl宏包}
\author{哈尔滨合正科技}
\date{2020年10月8日}

\begin{document}
\maketitle

\pagecolor{green!1!white}

\texttt{xcolor}是在\texttt{color}基础上扩展的颜色宏包,\texttt{colortbl}
用于给。察看文档:

\begin{verbatim}
texdoc color
texdoc xcolor
texdoc colortbl
\end{verbatim}

因为没有颜色方面的基础知识,这个宏包的文档较难理解。本文主要参考了刘海洋
撰写的《\LaTeX{}入门》的5.4节“使用彩色”(第365页至第369页)。

\section{color}

\texttt{color}宏包中,使用颜色的基本命令是\verb!\color!和
\verb!\textcolor!:

\begin{verbatim}
\color{<颜色>}
\textcolor{<颜色>}{<文字>}
\end{verbatim}

\verb!\color!属于声明式命令,它使后面的内容都使用指定的颜色输出,而
\verb!\textcolor!则将参数<文字>以指定的颜色输出。

示例:

{\color{red} 这些文字都会用红色显示。}
\textcolor{red}{这些文字也会用红色显示。}

其它命令包括:

\begin{verbatim}
\pagecolor{<页面颜色>}  % 改变页面颜色
\colorbox{<盒子颜色>}{<文字>} 
\fcolorbox{<线框颜色>}{<盒子颜色>}{<文字>}
\end{verbatim}

示例:

如本文档所示,\verb|\pagecolor{green!2!white}|改变了文档背景色。注意,这
里颜色的描述方式是在\texttt{xcolor}宏包中实现的。

\colorbox{lime}{lime}

\fcolorbox{black}{lime}{lime填充颜色加上black框}

\subsection{颜色模型}

所有的颜色都可采用某种所谓颜色模型来定义。一共有3种颜色模型:gray(灰度)、
rgb(红绿蓝)、cmyk(印刷四分色)。在使用颜色时,可指令颜色模型,然后使用特
定颜色模型下的分量[0, 1]之间的数值来表示具体的颜色。

分析下面源码:

\textcolor[gray]{0.5}{ 50\ \%灰色的文字}

{\color[rgb]{0.6,0.6,0} 暗黄色}

\subsection{预定义颜色}

预定义的颜色名见文档第48-40页,如:\mybox{black}、\mybox{lime}、
\mybox{pink}、\mybox{violet}、\mybox{orange}等。

\subsection{定义颜色}

\begin{verbatim}
\definecolor{<色彩名>}{<模型>}{<分量值>}
\end{verbatim}

紫色定义为:
\definecolor{myPurple}{cmyk}{0.45,0.85,0,0}
\mybox{myPurple}

\section{xcolor}

\begin{verbatim}
\usepackage[<options>]{xcolor}
\end{verbatim}

这里只介绍选项\texttt{特定预定义颜色名称集合},包
括:\texttt{dvipsnames}、\texttt{svgnames}、\texttt{x11names}。

\subsection{颜色表达式}

为了方便颜色定义,\texttt{xcolor}引入了颜色表达式的记法:

\begin{table}[H]
  \centering
  \begin{tabular}{ll}
    半色调& \verb|<颜色>!<百分数>|  \\
    混合色调& \verb|<颜色>!<百分数>!<颜色>|  \\
    互补色& \verb|-<颜色>|  \\
  \end{tabular}
\end{table}

示例:

\textcolor{purple!70}{淡紫色,70\%紫色}

{\color{blue!60!black}{60\%蓝色 与40 \%黑混合的深蓝色}}

\colorbox{-red}{\textcolor{red}{与红色互补的青色}}

\subsubsection{命名颜色表达式}

\verb!\colorlet!可以给颜色表达式命名。

示例:

\colorlet{darkred}{red!50!black}
\textcolor{darkred}{定义暗红色}

\section{colortbl}

\texttt{colortbl}用来实现彩色表格,可设置表列、表行、单元格或表格线的颜
色。

\verb!columncolor!示例:

\begin{table}[H]
  \centering
  \begin{tabular}{>{\columncolor{gray}}c >{\columncolor{lightgray}}c}
    深 & 浅 \\
    darker & lighter \\
  \end{tabular}
\end{table}

\verb!rowcolor!示例:

\begin{table}[H]
  \centering
  \begin{tabular}{>{\columncolor{gray}}c >{\columncolor{lightgray}}c}
    \hline \rowcolor{yellow}
    深 & 浅 \\
    darker & lighter \\
  \end{tabular}
\end{table}

\verb!cellcolor!示例:

\begin{table}[H]
  \centering
  \begin{tabular}{cccc}
    \hline
    No & No & \cellcolor{lightgray}Yes & No \\
   \cellcolor{lightgray} No & No & Yes   & No \\
    \hline
  \end{tabular}
\end{table}

当给\texttt{xcolor}宏包加上\texttt{table}选项时,它支持如下的命令:

\begin{verbatim}
\rowcolors[<横线命令>]{<起始行>}{<奇数行色彩>}{<偶数行色彩>}
\rowcolors*[<横线命令>]{<起始行>}{<奇数行色彩>}{<偶数行色彩>}
\end{verbatim}

这些命令用在\texttt{tabular}和\texttt{array}环境之前,可以让表格从<起始
行>开始,在奇数行和偶数行使用不同的背景色,同时在每一行前后执行可选的<横
线命令>画出横线,<横线命令>可以使用\verb!\hline!,这样每行表格会自动画出
横线。对于是带星号的版本,则<横线命令>只要交错景色的行执行。

示例:

\begin{table}[H]
  \centering
  \rowcolors{2}{black!20}{black!10}
  \begin{tabular}{crrr}
    \hline
    \rowcolor{black!30}
    项目 & 数值 & 数值 & 数值 \\
    A & 10 & 20 & 30 \\
    B & 10 & 20 & 30 \\
    C & 10 & 20 & 30 \\
    D & 10 & 20 & 30 \\
    E & 10 & 20 & 30 \\
    \hline
  \end{tabular}
\end{table}

\end{document}
%%% Local Variables:
%%% mode: latex
%%% TeX-master: t
%%% End:
