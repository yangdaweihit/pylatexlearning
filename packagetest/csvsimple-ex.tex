\documentclass{article}
\usepackage[UTF8]{ctex}
\usepackage{tabularx}
\usepackage{longtable}
\usepackage{tabu}
\usepackage{amsmath}
\usepackage{multirow}
\usepackage{fancyhdr}
\usepackage[top=3cm, bottom=3cm, left=3cm, right=3cm]{geometry}
\usepackage{float}
\usepackage{booktabs}
\usepackage{lastpage}
\usepackage[dvipsnames,table]{xcolor}
\usepackage{caption}% 解决表格标题与下方表格过近
\usepackage{csvsimple}
%%%表格标题与前后文字间距%%%%%%%%%%%%%%%%%
\setlength{\abovecaptionskip}{2pt} 
\setlength{\belowcaptionskip}{2pt}
%%%%%%%%%%%%%%%%%%%%%%%%%%%%%%%%%%%%%%

\pagestyle{fancy}
\fancyhead[C]{}
\fancyhead[L]{csvsimple宏包}
\fancyhead[R]{哈尔滨合正科技}
\fancyfoot[C]{\thepage\ / \pageref{LastPage}}
\renewcommand{\headrulewidth}{0.8pt}
\renewcommand{\footrulewidth}{0pt}

\title{csvsimple宏包}
\author{哈尔滨合正科技}
\date{2020年10月7日}

\begin{filecontents*}{grade.csv}
name,givenname,matriculation,gender,grade
Maier,Hans,12345,m,1.0
Huber,Anna,23456,f,2.3
Weisbaeck,Werner,34567,m,5.0
\end{filecontents*}

\begin{document}
\maketitle
% 在我的Emacs配置中:
% C-c RET 插入宏
% C-c C-e 插入环境
\texttt{csvsimple}宏包提供了一个处理CSV文件的界面,可支持数据过滤和表格
生成。查看该宏包可由终端执行:
\begin{verbatim}
texdoc csvsimple
\end{verbatim}


\section{介绍}

生成一个自动表格。

\begin{table}[H]
  \centering
  \csvautotabular{grade.csv}
\end{table}

\begin{table}[H]
  \centering
  \begin{tabular}{lc}
    \toprule%
    \bfseries Person & \bfseries Matr.~No.
    \csvreader[head to column names]{grade.csv}{}%
                       {\\\givenname\ \name & \matriculation}\\%
    \bottomrule
  \end{tabular}
\end{table}

% \begin{tabular}{l|c}%
%   % specify table head
%   \bfseries Person & \bfseries Matr.~No.
%   % use head of csv as column names
%   % specify your coloumns here
%   \csvreader[head to column names]{grade.csv}{}
%   {\\\hline\givenname\ \name & \matriculation}
% \end{tabular}

\section{选项}


\section{示例}


\end{document}
%%% Local Variables:
%%% mode: latex
%%% TeX-master: t
%%% End:
