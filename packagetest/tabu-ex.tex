\documentclass{article}
\usepackage[UTF8]{ctex}
\usepackage{tabularx}
\usepackage{longtable}
\usepackage{tabu}
\usepackage{amsmath}
\usepackage{multirow}
\usepackage{fancyhdr}
\usepackage[top=3cm, bottom=3cm, left=3cm, right=3cm]{geometry}
\usepackage{float}
\usepackage{booktabs}
\usepackage{lastpage}
\usepackage[dvipsnames,svgnames]{xcolor}
\usepackage{caption}% 解决表格标题与下方表格过近
\usepackage{csvsimple}
\usepackage{cprotect}% http://ctan.org/pkg/cprotect
%%%表格标题与前后文字间距%%%%%%%%%%%%%%%%%
\setlength{\abovecaptionskip}{2pt} 
\setlength{\belowcaptionskip}{2pt}
%%%%%%%%%%%%%%%%%%%%%%%%%%%%%%%%%%%%%%

\pagestyle{fancy}
\fancyhead[C]{}
\fancyhead[L]{\LaTeX{}表格}
\fancyhead[R]{哈尔滨合正科技}
\fancyfoot[C]{\thepage\ / \pageref{LastPage}}
\renewcommand{\headrulewidth}{0.8pt}
\renewcommand{\footrulewidth}{0pt}

\title{\LaTeX{}表格}
\author{哈尔滨合正科技}
\date{2020年10月6日}

\begin{document}
\maketitle
% 在我的Emacs配置中:
% C-c RET 插入宏
% C-c C-e 插入环境
为自动报告开发者在较短时间内了解\LaTeX{}各类表格环境,撰写本文档。文档从
基础到高级,包括了\texttt{tabular}、\texttt{taularx}、\texttt{longtable}、
\texttt{tabu}和\texttt{longtabu}。

本文适合结合代码讲解表格用法,仅浏览\texttt{pdf}文档无法了解表格的全部用
法。

\section{Tabular}

Tabular是\LaTeX{}原装的表格环境。后来功能更加强大的表格环境都是在Tabular
基础上逐步丰富起来的。要查看关于Tabular的手册,在终端键入:

\begin{verbatim}
  texdoc array
\end{verbatim}

你应该会感到奇怪:为什么不是\texttt{tabular},而是\texttt{array}。因为文
档的文件名是\texttt{array.pdf}。\texttt{tabular}和\texttt{array}是两个宏
包,它们的说明放在了一个文档中。

Tabular的环境用法为:

\begin{verbatim}
\begin{tabluar}[选项]{列式标记}
  <item1> & <item2> ... \\
  \hline
  <item1> & <item2> ... \\
\end{tabular}
\end{verbatim}
其中:\verb!&!用来分隔单元格;\verb!\\!用来换行;\verb!\hline!用来在在行间绘横线。

在\verb![ ]!中的选项只在需要时写入,选项包括:\texttt{t}、\texttt{b}。用
于进行竖向对齐。如:

\begin{tabular}{|c|}
center-\\ aligned \\
\end{tabular}
,
\begin{tabular}[t]{|c|}
center-\\ aligned \\
\end{tabular}
,
\begin{tabular}[b]{|c|}
center-\\ aligned \\
\end{tabular}

\subsection{列式标记}

Tabular最复杂的是\textbf{列式标记},这一部分是必须填写的参数。常见的列式
标记见下表。

\begin{table}[H]
  \centering
  \caption{列式标记}
  \label{tab:col-spec}
  \begin{tabular}{ll}
    \toprule
    \textbf{列格式}  & \textbf{说明} \\
    \midrule
    \verb!l/c/r!   & 单元格内容左对齐/居中/右对齐,不折行 \\
    \verb!p{<width>}!  & 单元格固定宽度<width>,可自动折行 \\
    \verb!|!       & 绘制竖线 \\
    \verb!@{<string>}!  & 自定义内容<string> \\
    \bottomrule
  \end{tabular}
\end{table}

下面举个例子:

\begin{table}[H]
  \centering
  \begin{tabular}{lcr|p{6em}}
    \hline
    left & center & right & par box with fixed width\\
    L & C & R & P \\
    \hline
  \end{tabular}
\end{table}

表格中每行的单元格数目不能多于列格式里\texttt{l/c/r/p}的总数(可以少于这
个总数),否则出错。

\texttt{@} 格式可在单元格前后插入任意的文本,但同时它也消除了单元格前后额
外添加的间距。\texttt{@}格式可以适当使用以充当“竖线”。特别
地,\texttt{@{}} 可直接用来消除单元格前后的间距:

\begin{table}[H]
  \centering
  % 在第一列左侧插入一个"空",消除了默认的间距
  % 在第一列的右侧插入一个冒号
  % 在第三列的右侧插入一个“空”,消除了默认的间距
  \begin{tabular}{@{}l @{:}l  r@{}}
    \hline
    1 & 1 & one \\
    11 & 3 & eleven \\
    \hline
  \end{tabular}
\end{table}

在列式标记中,还提供了一种简写方式:\verb!*{<n>}{<列式标记>}!,比如下面两
种写是等效的:

\begin{verbatim}
\begin{tabular}{|c|c|c|c|c|p{4em}|p{4em}|}
\begin{tabular}{|*{5}{c|}*{2}{p{4em}|}}
\end{verbatim}

\texttt{array}宏包提供了一个辅助格式\texttt{>}和\texttt{<},用于给列格式
前后加上修饰命令。
   
\begin{table}[htpb]
  \centering
  % 给第一列开始加了一个修饰,即\itshape
  % 给第一列后面加了一个修饰,加上了一个*
  \begin{tabular}{ >{\itshape}l<{*}   r}
    \hline
    italic & normal \\
    column & column \\
    \hline
  \end{tabular}
\end{table}

\begin{table}[H]
  \centering
  % 给第一列开始加了一个修饰,即$
  % 给第一列后面加了一个修饰,加$
  % 这样第一列就可以输入数学公式了
  \begin{tabular}{>{$}r<{$ --} @{\ }l}
    % \hline
    \alpha & normal \\
    \sqrt{2} & column \\
    % \hline
  \end{tabular}
\end{table}

同理,还可以通过辅助修饰符,实现对固定宽度列对齐方式的定义,如:

\begin{table}[H]
  \centering
  % 给第一列开始加了一个修饰,即$
  % 给第一列后面加了一个修饰,加$
  % 这样第一列就可以输入数学公式了
  % 一般还要加额外的命令 \arraybackslash 以免出错
  % \centering等对齐命令会破坏表格环境里 \\换行命令的定
  % 义,\arraybackslash 用来恢复之。
  \begin{tabular}{ | >{\centering\arraybackslash}p{8em} | r |}
    \hline
    center & normal \\
    \hline
    text & column \\
    \hline
  \end{tabular}
\end{table}

\texttt{array}宏包还提供了类似 p 格式的 m 格式和 b 格式,三者分别在垂直方
向上靠顶端对齐、居中以及底端对齐。

\newcommand{\txt}{a b c d e f g h i}

\begin{table}[H]  \centering
  \begin{tabular}{cp{2em}m{2em}b{2em}}
    \hline
    pos & \txt & \txt  & \txt \\ \hline
  \end{tabular}
\end{table}

\texttt{Tabular}新增了一对格式:\verb!w{align}{width}!和
\verb!W{align}{width}!,用于同时定义宽度和对齐方式。\texttt{W}的特殊之
处在于,当内容不适合时,会弹出警告。

\begin{table}[H]
  \centering
  \begin{tabular}{|c|w{r}{6em}|w{c}{6em}|W{l}{2em}|} \hline pos &
    righ-aligned & center & left \\ \hline
  \end{tabular}
\end{table}

关于\texttt{Tabular}的列格式定义还有一些,详见\texttt{array.pdf}。

\subsection{横线}

我们已经在之前的例子见过许多次绘制表格线的\verb!\hline!命令。另外 \verb!\cline{⟨i⟩-⟨j⟩}!用来
绘制跨越部分单元格的横线:  

\begin{table}[h!]
  \centering
  \begin{tabular}{|*{3}{c|}}
    \hline
    4 & 9 & 2 \\ \cline{2-3}
    3 & 5 & 7 \\ \cline{1-1}
    8 & 1 & 6 \\ \hline
  \end{tabular}
\end{table}

在科技论文排版中广泛应用的表格形式是三线表,形式干净简明。三线表由\texttt{booktabs}宏包
支持,它提供了 \verb!\toprule!、\verb!\midrule! 和 \verb!\bottomrule! 命令用以排版三线表的三条线,以及和
\verb!\cline! 对应的\verb!\cmidrule!。除此之外,最好不要用其它横线以及竖线:

\begin{table}[H]
  \centering
  \begin{tabular}{cccc}
    \toprule
    & \multicolumn{3}{c}{Numbers} \\
    \cmidrule{2-4}
    & 1 & 2 & 3 \\
    \midrule
    Alphabet & A & B & C \\
    Roman & I & II& III \\
    \bottomrule
  \end{tabular}
\end{table}

\subsection{行距控制}

有两种方式可以改变表格的行间距。
\begin{itemize}
\item 修改参数 \verb!\arraystretch!可以得到行距更加宽松
的表格
\item 换行命令\verb!\\! 添加可选参数
\end{itemize}

如下表示例:
  
\renewcommand\arraystretch{1.8}
\begin{table}[H]
  \centering
  \begin{tabular}{|c|}
    \hline
    Really loose \\  \hline
    Really loose \\ [6pt] \hline
    tabular rows.\\ \hline
  \end{tabular}
\end{table}
\renewcommand\arraystretch{1}
\section{tabularx}

\texttt{tabularx}宏包引入一个新的选项\texttt{X}。使得表格总宽度可以先定义,
然后自动计算\verb!X!定义的列宽度。

要察看\texttt{tabuarlx}的文档,在终端键入:
\begin{verbatim}
  texdoc tabularx
\end{verbatim}

看个例子:

\begin{table}[h!]
  \centering
  % 除了第一列是按实际所需宽度定义外
  % 其它两列按剩余宽度平分
  \begin{tabularx}{0.8\textwidth}{|c|*{2}{X|}}
    \hline
    a &  b & c \\
    \hline
    aaaaa &  b & c  \\
    \hline
  \end{tabularx}
\end{table}

为了将上表中的后两列也居中对齐,仍然需要按\texttt{tabular}环境列格式的辅
助修饰符做定义。

\begin{table}[h!]
  \centering
  % 除了第一列是按实际所需宽度定义外
  % 其它两列按剩余宽度平分
  \begin{tabularx}{0.8\textwidth}{|c|>{\centering\arraybackslash}X|X|}
    \hline
    a &  b & c \\
    \hline
    aaaaa &  b & c  \\
    \hline
  \end{tabularx}
\end{table}

\section{longtable}

长表格宏包\texttt{longtable}相当于\texttt{tabular}在能力上的扩展,即具备
了跨页能力。

需要注意的是,它将浮动环境\texttt{table}的功能包括了进来,即拥有
了\verb!\caption!命令,还可以和\texttt{table}一样被\verb!\listoftabls!命
令列入表格目录中。这意味着,你不应该将\texttt{longtable}环境置于
\texttt{table}环境中,它本身就构成了一个浮动体。

\texttt{longtable}手册用一个长表格实例展示了它的所有功能。这里也给出一个
完整的示例。

\begin{longtable}{@{*}r||p{2in}@{*}}

  \caption{长表实例}\label{tbl:longtable} \\
  \toprule
  首列    &  次列  \\
  \midrule
  \endfirsthead
  
  \multicolumn{2}{c}{表\ref{tbl:longtable}: 长表实例(续表)}\\%
  \toprule
  \textbf{首列}    &  \textbf{次列}  \\
  \midrule
  \endhead

  \midrule
  \multicolumn{2}{r}{续表见下页}\\%
  \endfoot
  
  \bottomrule
  \endlastfoot

  longtable columns are specificed & in the \\
  same way as in the tabular       & environment. \\
  \verb!{@{*}r||p{1in}@{*}}!          & in this case. \\
  Each row ends with a             & \verb!\\! command. \\
  The \verb!\\! command has an      & 可选  \\
  参数,如同在                    & \texttt{tabular} 环境中。 \\
  看到[10pt]的效果了吗              & ?   \\  [10pt]
  很多行                            & 象这样。 \\
  很多行                            & 象这样。 \\
  很多行                            & 象这样。 \\
  很多行                            & 象这样。 \\
  很多行                            & 象这样。 \\
  很多行                            & 象这样。 \\
  很多行                            & 象这样。 \\
  很多行                            & 象这样。 \\
  很多行                            & 象这样。 \\
  很多行                            & 象这样。 \\
  很多行                            & 象这样。 \\
  很多行                            & 象这样。 \\
  很多行                            & 象这样。 \\
  很多行                            & 象这样。 \\
  很多行                            & 象这样。 \\
  很多行                            & 象这样。 \\
  很多行                            & 象这样。 \\
  很多行                            & 象这样。 \\
  很多行                            & 象这样。 \\
  很多行                            & 象这样。 \\
  很多行                            & 象这样。 \\
  很多行                            & 象这样。 \\
  很多行                            & 象这样。 \\
  很多行                            & 象这样。 \\
  很多行                            & 象这样。 \\
  很多行                            & 象这样。 \\
  很多行                            & 象这样。 \\
  很多行                            & 象这样。 \\
  很多行                            & 象这样。 \\
  来一个\verb!\hline!            & 。   \\
  \hline
  来两个\verb!\hline\hline!   & 。 \\
  \hline\hline 
  \multicolumn{2}{|c|}{这是跨两列的效果} \\
    选项\verb![t] [b] [c]!    & 不能再被使用。  \\
    \verb![l] [r] [c]! 中的选项 &  之一可被使用。 \\
\end{longtable}
  
\section{tabu和longtabu}


\texttt{tabu}宏包极大了扩展了表格的能力,\texttt{longtabu}是其跨页长表版
本。\texttt{tabu}的功能十分强大,让人意外的是,它的作者已经失联。最新的
版本停留在2011年2月16日的2.8。

先看几个个简单的示例。

\begin{table}[H]
 \centering
 \begin{tabu} to 80mm {|X[1,l] | X[2,c] | X[3,c] | X[1,r] | }
 \hline
 |\dotfill | & Text  & Text & Text \\ 
 \hline
 Text & Text & Text & Text  \\
 \hline
 \end{tabu}
\end{table}

\begin{table}[H]
  \centering
  \tabulinestyle {3pt ForestGreen}
  \begin {tabu}{|X[m]|X[cm]|}
    \tabucline - 
    \begin{tabu}{X[$]X[$]}
    \alpha & \beta \\
    \gamma & \delta + \epsilon + \zeta + \eta + \theta
    \end{tabu}
    &
    This is a tabu with negativ width coefficients for \texttt{X} columns
    \\ \tabucline -
  \end{tabu}
\end{table}


\subsection{\texttt{tabu}环境}

\subsubsection{tabu,tabu to 和 tabu spread}
\begin{verbatim}
\begin {tabu} [pos] {tabular preamble}
\begin {tabu} to <dimen> [pos] {tabular preamble}
\begin {tabu} spread <dimen> [pos] {tabular preamble}
\end{verbatim}

\texttt{tabu}环境和\texttt{tabular}行为基本一致,改进之处包括:
\begin{itemize}
\item 在\texttt{tabu}中可以使用\textbf{footnotes};
\item \texttt{X}列带有选项,包括:宽度系数、对齐定义(r,c,l或R,C,L,
  J)、列类型(p,m,b);
\item 可在\texttt{tabu}中同时包含文字和公式;
\item \texttt{tabu}中的表格中可以包含\texttt{tabular}、\texttt{tabular*}、
  \texttt{tabularx}或\texttt{tabu};反之,\texttt{tabu}可包含于
  \texttt{tabular}、\texttt{tabularx}之中;
\item \texttt{tabu}和很多宏包兼容,可以使用\verb!\raggedleft!、
  \verb!\raggedright!或\verb!\centering!,且不必考虑
  \verb!\arraybackslash!,而\verb!\\!也可以正常在\texttt{X}列中使用。
\end{itemize}

\verb!\begin{tabu} to <dimen>! 定义了表格整体的宽度。

\verb!\begin{tabu} spread <dimen>! 定义了表格在自然宽度外增加\verb!<dimen>!。

\subsubsection{longtabu,longtabu to 和 longtabu spread}
\begin{verbatim}
\begin {longtabu} [l | c | r] {tabular preamble}
\begin {longtabu} to <dimen> [l | c | r] {tabular preamble}
\begin {longtabu} spread <dimen> [l | c | r] {tabular preamble}
\end{verbatim}

\texttt{longtabu}就是可跨页的\texttt{tabu}。它是基于\texttt{longtable}的,
所以使用时必须加载\texttt{longtable}宏包。由此,\verb!\endhead!、
\verb!\endfirsthead!、\verb!\endfoot!、\verb!\endlastfoot!或
\verb!\caption!均可在\texttt{longtabu}中使用。

\texttt{longtabu}相对于\texttt{longtable}增强之处在于,可使用\texttt{X}
列,以及对于水平和竖直线的线定义。

在下表源码中,注意计数器及\texttt{csvexample}的使用。

\newcounter{elio}
\setcounter{elio}{0} 
\begin{longtabu} to 100mm {X[3, l] X[3, l] X[3, c] X[2, c] X[2, r] }

  \caption{longtabu长表实例}\label{tbl:longtabu} \\

  \toprule
  名 & 姓 & 入学考试 & 性别  &  年级   \\
  \midrule
  \endfirsthead

  \multicolumn{5}{c}{
  \refstepcounter{elio}
  表 \ref{tbl:longtabu}:longtabu长表实例 (续表\theelio{})} \\
  \toprule
  名 & 姓 & 入学考试 & 性别  &  年级   \\
  \midrule
  \endhead
  
  \midrule
  \multicolumn{5}{r}{续表\theelio{}见下页}\\%
  \endfoot
  
  \bottomrule
  \endlastfoot
  \csvreader[late after line=\\]{data.csv}{
    name=\name,
    givenname=\givenname,
    matriculation=\matriculation,
    gender=\gender,
    grade=\grade
  }{\name & \givenname &\matriculation & \gender & \grade}
\end{longtabu}


\subsubsection{X列——控制水平空间}

\texttt{tabu X}列可视为\texttt{taluarx X}列的增强版,但彼此不能相互作用。

\begin{itemize}
\item \textbf{宽度系数}为\texttt{X}列的可选项。如\verb!X[2.5]X[1]!,等同
  于\verb!X[2.5]X!、\verb!X[5]X[2]!;
\item \textbf{负宽度系数}为\texttt{X}列的可选项。如\verb!X[-2.5]X[1]!,等同
  于\verb!X[-2.5]X!、\verb!X[-5]X[2]!;

  负宽度指的是首列\textbf{顶多}是第二列的是2.5倍。如果首列自然宽度小于第
  二列宽的2.5倍,则会缩至其自然宽度。
\item 水平对齐定义更为容易,如\verb!X[5,r]|X[2,c]!。竖向对齐也可定义,如
  \verb!X[5,r,m]X[2,p,c]!,其中逗号可省略。
\item \texttt{tabu X}列可由\verb!\multicolumn!跨列;
\end{itemize}

关于\texttt{tabu spread}的效果见\texttt{tabu.pdf}第11页。

\cprotect\subsubsection{\verb!\tabuphantomline!}

在\texttt{tabu}中使用\verb!\multicolumn!列时,需要表格结尾处加上命
令\verb!\tabuphantomline!。原因没看懂,在手册中的第12页。

在\verb!\tabuphantomline!之后,不得跟
上\verb!\cr!或\verb!\\!或\verb!\tabularnewline!。见下面的示例:

\begin{tabu}{|X|X|X[2]|}
  \tabucline-
  a & b & c \\
  \tabucline-
  \multicolumn{2}{|c|}{Hello} & World \\
  \tabucline-
  \tabuphantomline  %注意此处要加上这个命令
\end{tabu}

\cprotect\subsubsection{\verb!\tabulinesep!和\verb!\extrarowsep!—控制竖向
  空间}

示例见文档第13页。

\subsection{线引导或颜色}

\subsubsection{竖线:|有一个选项参数}

见下表源码,竖线后面可以跟线宽及颜色选项:

\begin{table}[H]
  \setlength\doublerulesep{1pt}
  \centering
  \begin{tabu} to 0.8\textwidth {| |[1.5pt red] |[0.5pt blue] X[c] | X[c] | [0.5pt] |
      [1.5pt ]} 
    Hello & World \\
    Hello & World \\
    \hline
  \end{tabu}
\end{table}

\cprotect\subsubsection{多个\verb!\firsthline!和\verb!\lasthline!}

这两个命令的用途没太看懂,暂时搁置。

\subsubsection{更多样式}

\begin{verbatim}
\taburulecolor {<rule color>}
\taburulecolor |<double rule sep color>|{<rule color>}
\end{verbatim}

\verb!\taburulecolor!用来定义线的颜色,包括:\verb!\hline!、
\verb!\firsthline!、\verb!\lasthline!,以及竖线。可改变竖线颜色,需激活
所谓标准线,方法是使用命令\verb!\tabulinestyle{}!或\verb!\tabureset!。

命令中两条竖线中的可选参数表示两条线间的颜色。若没有,则不填充颜色。

见下表源码,分析命令功能。

\begin{table}[H]
  \centering
  \taburulecolor|lime|{DarkSlateBlue}
  \arrayrulewidth =1mm \doublerulesep=2mm
  \begin{tabu} spread 0pt {|c|}
  \firsthline \hline
  a tabu \\
  environment \\
  made with \\
  \lasthline[5mm] \hline \hline
  \end{tabu} 
\end{table}

\begin{verbatim}
\tabulinestyle {hline style specificationi}
\end{verbatim}

\verb!\tabulinestyle!定义竖线和水平线的样式。

体会下表源码中\verb!\tabulinestyle!的用法。

\begin{table}[H]
  \centering
  \taburulecolor|lime|{DarkSlateBlue}
  % 线宽是3pt;实线蓝色,长度3pt;断线无色,长度为1pt
  \tabulinestyle{3pt on 3pt blue off 1pt }
  \begin{tabu} spread 0pt {|c|}
  \firsthline \hline
  a tabu \\
  environment \\
  made with \\
  \lasthline[5mm] \hline \hline
  \end{tabu} 
\end{table}

\begin{verbatim}
\tabucline [style or spec.]{start-end}
\end{verbatim}

\verb!\tabucline!定义了局部线的样式和起止列。例子见文档第17页。它的功能
很强大,可定义线的任意样式,详见文档中的示例。

\subsubsection{自动水平线}

\begin{verbatim}
\everyrow{code}
\end{verbatim}

\verb!everyrow!用于自动插入水平线。见下表示例。

\begin{table}[H]
  \centering
  \begin{tabu} to .5\linewidth {cX[2mc]X}
    \tabucline[1pt] -
    \everyrow{\tabucline[on 2pt] -}
    This i s & a small example &of a \texttt{tbu} \\
    which &automatically &inserts \\
    a horizontal &line after &each of its row \everyrow{} \\
    \tabucline[1pt] -
  \end{tabu}
\end{table}

\subsection{修改一行内的字体和对齐格式}

\begin{verbatim}
\rowfont [alignment]{font specification}
\end{verbatim}

\verb!\rowfont!可以设置一整行文字的字体。同时还可以设置该行内的对齐方式。

参见源码,体会下表功能。

\begin{table}[H]
  \centering
  \begin{tabu}{|X|X|}
    \tabucline-
    % \rowfont[c]{\bfseries}  %设置首行为居中对齐 c选项不好使
    % \rowfont{\centering \bfseries}  %设置首行为居中对齐,代之以\centering
    \rowfont[c]{\bfseries}  %设置首行为居中对齐,代之以\centering
    This  & Is  \\ \tabucline[on 2pt,blue]-
    % 第二行按默认左对齐
    tabu  & package  \\ \tabucline[off 2pt blue]-
    % 第三行设置为右对齐
    \rowfont[r]{\itshape}
    for &\texttt{tabu} and \texttt{longtabu} \\\tabucline-
  \end{tabu}
\end{table}

\end{document}
%%% Local Variables:
%%% mode: latex
%%% TeX-master: t
%%% End:
