\documentclass{article}
\usepackage[UTF8]{ctex}
\usepackage{tabularx}
\usepackage{longtable}
\usepackage{tabu}
\usepackage{amsmath}
\usepackage{multirow}
\usepackage{fancyhdr}
\usepackage[top=3cm, bottom=3cm, left=3cm, right=3cm]{geometry}
\usepackage{float}
\usepackage{booktabs}
\usepackage{lastpage}
\usepackage[dvipsnames,svgnames]{xcolor}
\usepackage{caption}% 解决表格标题与下方表格过近
\usepackage{csvsimple}
\usepackage[colorlinks=true]{hyperref}
\usepackage{cprotect}% http://ctan.org/pkg/cprotect
\usepackage{graphicx}
%%%表格标题与前后文字间距%%%%%%%%%%%%%%%%%
\setlength{\abovecaptionskip}{2pt} 
\setlength{\belowcaptionskip}{2pt}
%%%%%%%%%%%%%%%%%%%%%%%%%%%%%%%%%%%%%%

\newcommand*{\mylogo}[1]{\begin{minipage}[c][2em][c]{2em}
 \includegraphics[width=1.5em]{#1}
  \end{minipage}
}

\pagestyle{fancy}
\fancyhead[C]{}
\fancyhead[L]{minipage宏包}
\fancyhead[R]{哈尔滨合正科技 \mylogo{logo.png}}
\fancyfoot[C]{\thepage\ / \pageref{LastPage}}
\renewcommand{\headrulewidth}{0.8pt}
\renewcommand{\footrulewidth}{0pt}

\title{minipage宏包}
\author{哈尔滨合正科技}
\date{2020年10月8日}

\begin{document}
\maketitle

\texttt{minipage}是一个经常被使用到的环境。但它的文档目前是用德文书写的。
本文内容来源于:\url{http://joshua.smcvt.edu/latex2e/minipage.html}。

\section{语法}

环境语法为:

\begin{verbatim}
\begin{minipage}[position][height][inner-pos]{width}
  contents
\end{minipage}
\end{verbatim}

\texttt{position}有3个选项:
\begin{description}
\item[c] 居中对齐;
\item[t] 顶对齐;
\item[b] 底对齐;
\end{description}

\texttt{height}是一个长度参数。

\texttt{inner-pos}是内部对齐参数,有4个选项:

\texttt{position}有4个选项:
\begin{description}
\item[c] 居中对齐;
\item[t] 顶对齐;
\item[b] 底对齐;
\item[s] Sretch contents out vertically; it must contain vertically
  stretchable space. 这句话没看懂。
\end{description}

\section{示例}

示例(为了显示minipage的大小,将其插入到了一个tabular表格中):

\vskip1em
\begin{tabular}{|c|}
  \hline
  ---
\begin{minipage}[t][2in][c]{1.2in}
  2inx2in 的minipage\\居中对齐的效果
\end{minipage}--- \\
  \hline
\end{tabular}
\begin{tabular}{|c|}
  \hline
  ---
\begin{minipage}[c][2in][c]{1.2in}
  2inx2in 的minipage\\居中对齐的效果
\end{minipage}--- \\
  \hline
\end{tabular}
\begin{tabular}{|c|}
  \hline
  ---
\begin{minipage}[b][2in][c]{1.2in}
  2inx2in 的minipage\\居中对齐的效果
\end{minipage}--- \\
  \hline
\end{tabular}
\vskip1em

默认情况下,\texttt{minipage}中的段落起始是没有缩进的。可设置缩进可以在
内容开始处设置:

\verb!\setlength{\parindent}{1pc}!

\vskip1em
\begin{tabular}{|c|}
  \hline
  ---
\begin{minipage}[b][2in][c]{1.2in}
  \setlength{\parindent}{1pc}
  居中对齐的效果
  居中对齐的效果
  居中对齐的效果
  居中对齐的效果
\end{minipage}--- \\
  \hline
\end{tabular}
\vskip1em

脚注在\texttt{minipage}中是有效的。因为脚注在插图或表格中特别有用。脚注
显示在\texttt{minipage}中的底部,而不是页面的底部。

\begin{center}           % center the minipage on the line
\begin{minipage}{2.5in}
  \begin{center}         % center the table inside the minipage
    \begin{tabular}{ll}
      \textsc{Monarch}  &\textsc{Reign}             \\ \hline
      Elizabeth II      &63 years\footnote{to date} \\
      Victoria          &63 years                   \\
      George III        &59 years
    \end{tabular}
  \end{center}  
\end{minipage}
\end{center}

体会一下下面的例子。在一个插图右侧,以居中对齐的方式放置了一个
\texttt{minipage}环境,然而在里面放置了一个表格。

注意,\texttt{minpage}里面是不能包含浮动体的。

\newcommand*{\vcenteredhbox}[1]{\begin{tabular}{@{}c@{}}#1\end{tabular}}
\begin{center}
  \vcenteredhbox{\includegraphics[width=0.3\textwidth]{logo.png}}
  \hspace{0.1\textwidth}
  \begin{minipage}{0.5\textwidth}
    \begin{tabular}{r|l}
      \multicolumn{1}{r}{Borough} &Pop (million)  \\ \hline
      The Bronx      &$1.5$  \\
      Brooklyn       &$2.6$  \\
      Manhattan      &$1.6$  \\
      Queens         &$2.3$  \\
      Staten Island  &$0.5$  
    \end{tabular}
  \end{minipage}              
\end{center}

\section{后记}

为了显示\texttt{minipage}的作用,将本文档页眉的右部插入了一个图片。参见
源码32行。

本文测试中表格中的脚注时,发现页码中的最后页码变成线了红色。该问题暂时搁
置起来。

\end{document}
%%% Local Variables:
%%% mode: latex
%%% TeX-master: t
%%% End:
