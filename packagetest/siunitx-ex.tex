\documentclass{article}
\usepackage[UTF8]{ctex}
\usepackage{tabularx}
\usepackage{longtable}
\usepackage{tabu}
\usepackage{amsmath}
\usepackage{multirow}
\usepackage{fancyhdr}
\usepackage[top=3cm, bottom=3cm, left=3cm, right=3cm]{geometry}
\usepackage{float}
\usepackage{booktabs}
\usepackage{lastpage}
\usepackage[dvipsnames,table]{xcolor}
\usepackage{caption}% 解决表格标题与下方表格过近
\usepackage{siunitx}
\usepackage{fancyvrb}
\usepackage{cprotect}% http://ctan.org/pkg/cprotect
%%%表格标题与前后文字间距%%%%%%%%%%%%%%%%%
\setlength{\abovecaptionskip}{2pt} 
\setlength{\belowcaptionskip}{2pt}
%%%%%%%%%%%%%%%%%%%%%%%%%%%%%%%%%%%%%%

\pagestyle{fancy}
\fancyhead[C]{}
\fancyhead[L]{\texttt{siunitx}宏包}

\fancyhead[R]{哈尔滨合正科技}
\fancyfoot[C]{\thepage\ / \pageref{LastPage}}
\renewcommand{\headrulewidth}{0.8pt}
\renewcommand{\footrulewidth}{0pt}
\makeatletter

\newcolumntype{V}[1]{>{\topsep=0pt\@minipagetrue}p{#1}<{\vspace{-\baselineskip}}}
\makeatother
\newcommand{\cs}[1]{\texttt{\symbol{`\\}#1}}

\title{\texttt{siunitx}宏包}
\author{哈尔滨合正科技}
\date{2020年10月7日}

\begin{document}
\maketitle

\texttt{siunitx}宏包用于数字和单位的表达。

\begin{table}[H]
  \centering
  \caption{\texttt{siunitx}命令示例} 
  \begin{tabu*} to \linewidth {X[l,$$] X[2, l] }
    \toprule
    \num{12345,67890}  & \Verb!\num{12345,67890}!   \\
    \num{1+-2i}  & \Verb!\num{1+-2i}!   \\
    \num{.3e45}  & \Verb!\num{.3e45}!   \\
    \num{1.654 x 2.34 x 3.430} & \Verb!\num{1.654 x 2.34 x 3.430}! \\
    \si{kg.m.s^{-1}} & \Verb!\si{kg.m.s^{-1}}!  \\
    \si{\kilogram\metre\per\second}  &
    \Verb!\si{\kilogram\metre\per\second}! \\
    \si[per-mode=symbol]{\kilogram\metre\per\second} &
    \Verb!\si[per-mode=symbol]{\kilogram\metre\per\second}!  \\
    \numlist{10;20;30}  & \Verb!\numlist{10;20;30}! \\
    \SIlist{0.13;0.67;0.80}{\milli\metre} &
    \Verb!\SIlist{0.13;0.67;0.80}{\milli\metre}! \\
    \numrange{10}{20} & \Verb!\numrange{10}{20}! \\
    \SIrange{0.13}{0.67}{\milli\metre} &
    \Verb!\SIrange{0.13}{0.67}{\milli\metre}!  \\
    \bottomrule
    \end{tabu*}
\end{table}

本文先介绍宏包命令,再说明选项用法。

\section{命令}

\subsection{数字}

\begin{verbatim}
  \num[<options>]{<number>}
\end{verbatim}

\cprotect\subsubsection{\verb!\num!}

数字被\verb!\num!宏自动格式化。格式化去除数字中的软空格和硬空格,并自动
识别数字中的成份(默认标记用\texttt{e},\texttt{E},\texttt{d}或
\texttt{D}),在小数点面前自动加0,\texttt{'.'}和\texttt{','}被识别为小数点。

示例:


\begin{table}[H]
  \centering
  \begin{tabu*} to 0.5\linewidth {X[l,$$] X[l] }
    % \toprule
    \num{123}  & \Verb!\num{123} !   \\
    \num{0.123}  & \Verb!\num{0.123} !   \\
    \num{.1234}  & \Verb!\num{.1234} !   \\
    \num{3.45d-4}  & \Verb!\num{3.45d-4}!   \\
    \num{-e45}  & \Verb!\num{-e10}!   \\
    % \bottomrule
    \end{tabu*}
\end{table}

\cprotect\subsubsection{\verb!\numlist!}

数字列表可用\verb!\numlist!函数处理。参数为一个数字列表。

\begin{table}[H]
  \centering
  \begin{tabu*} to 0.5\linewidth {X[l,$$] X[l] }
    % \toprule
    \numlist{10;30;50;70}  & \Verb!\numlist{10;30;50;70} !   \\
    % \bottomrule
    \end{tabu*}
\end{table}


\cprotect\subsubsection{\verb!\numrange!}

\verb!\numrange!处理数据域,数字本身和\verb!\num!一样。
\begin{table}[H]
  \centering
  \begin{tabu*} to 0.5\linewidth {X[l,$$] X[l] }
    % \toprule
    \numrange{10}{30}  & \Verb!\numrange{10}{30} !   \\
    % \bottomrule
    \end{tabu*}
\end{table}


\cprotect\subsubsection{\verb!\ang!}

\verb!\ang!表达角度。
\begin{table}[H]
  \centering
  \begin{tabu*} to 0.5\linewidth {X[l,$$] X[l] }
    % \toprule
    \ang{10}  & \Verb!\ang{10}!   \\
    \ang{12.3}  & \Verb!\ang{12.3}!   \\
    \ang{4,5}  & \Verb!\ang{4,5}!   \\
    \ang{1;2;3}  & \Verb!\ang{1;2;3}!   \\
    \ang{;;1}  & \Verb!\ang{;;1}!   \\
    \ang{+10;;}  & \Verb!\ang{+10;;}!   \\
    \ang{-0;1;}  & \Verb!\ang{-0;1;}!   \\
    % \bottomrule
    \end{tabu*}
\end{table}

\subsection{单位}

\cprotect\subsubsection{\verb!\si!}

单位符号可由\verb!\si!命令排版。
\begin{table}[H]
  \centering
  \begin{tabu*} to 0.5\linewidth {X[l,$$] X[l] }
    % \toprule
    \si{kg.m/s^2}  & \Verb!\si{kg.m/s^2}!   \\
    \si{g_{polymer}~mol_{cat}.s^{-1}} &
    \Verb!\si{g_{polymer}~mol_{cat}.s^{-1}}! \\
    \end{tabu*}
\end{table}

\cprotect\subsubsection{\verb!\SI!}

\cprotect\subsubsection{\verb!\SIlist!}

\cprotect\subsubsection{\verb!\SIrange!}

\subsection{单位宏}

\texttt{siunitx}宏包手册第9页显示了常用单位的命令宏。示例如下表:

\begin{table}[H]
  \centering
  \begin{tabu*} to 0.5\linewidth {X[l,$$] X[l] }
    % \toprule
    \si{\degreeCelsius}  & \Verb!\si{\degreeCelsius}!   \\
    \si{\newton}  & \Verb!\si{\newton}!   \\
    \si{\ohm}  & \Verb!\si{\ohm}!   \\
    \si{\percent}  & \Verb!\si{\percent}!   \\
    \si{\radian}  & \Verb!\si{\radian}!   \\
    \si{\degree}  & \Verb!\si{\degree}!   \\
    \end{tabu*}
\end{table}

\subsection{建立新宏}

宏包提供了定义新单位宏的命令,此处从略。


\subsection{表格}

宏包为表格提供了一个\texttt{S}列,用于对齐内容。
\begin{table}[H]
\caption{\texttt{S}列的标准行为}
\label{tab:S:standard}
\centering
\begin{tabular}{S[
  group-four-digits=true,
  round-mode=places,
  round-precision=3,
  round-integer-to-decimal=true
  ]}
\toprule
{Some Values} \\
\midrule
  2.3456 \\
  34.2345 \\
  -6.7835 \\
  90.473 \\
  5642.5 \\
  0 \\
  1.2e3 \\
  e4 \\
\bottomrule
\end{tabular}
\end{table}


\begin{table}[H]
\caption{Detection of surrounding material in an \texttt{S}
column.}
\label{tab:S:extras}
\centering
\begin{tabular}{S[color=orange]}
\toprule
{Some Values} \\
\midrule
12.34 \\
\color{purple} 975,31 \\
44.268 \textsuperscript{\emph{a}} \\
\bottomrule
\end{tabular}
\end{table}

\section{选项}

\subsection{键-值控制系统}

\texttt{siunitx}包的行为由很多键-值选项控制。它们可以通过\verb!\sisetup!
进行全局设置,亦可以用户宏中的选项中设置。

键类型包括:选择、整数、长度、文字、宏、数学公式、元参数、开关。具体解释
见文档第17页。

\subsection{检查字体}

由宏包解析的单位或数字在打印时,我们可能需要使其打印结果与前面文字的字族、字形相符合。可通
过检查字体选项实现这一功能。包
括:\verb!detect-weight!、\verb!detect-family!、\verb!detect-shape!、
\verb!detect-mode!等。

\sisetup{
detect-family = true, %检查字族
detect-inline-family = math %检查行内数学公式中的字体
}%


\begin{table}[H]
  \centering
  \begin{tabu*} to 0.5\linewidth {X[l] X[l] X[l]}
    有衬线 $\num{1234}$  & \Verb!$\num{1234}$! & \\
    {\sffamily 无衬线 $\num{1234}$ }   & \Verb!{\sffamily $\num{1234}$}! \\
   $ \mathsf {\num{1234}} $ & \Verb!$\mathsf{\num{1234}} $! \\
     \sisetup{detect-inline-family = text} %检测数学公式中文字的字体
     { \sffamily 无衬线 $\num{1234}$ } & \Verb!{\sffamily 无衬线 $\num{1234}$ }!\\
 $ \mathsf { \num{1234} } $ & \Verb!$\mathsf{\num{1234} } $! \\
    \sisetup{
    detect-weight = true,
    detect-inline-weight = math
    }%
 $\num{5678}$ & \Verb!$\num{5678}$! \\
 {\boldmath $\num{5678}$ } & \Verb!{\boldmath $\num{5678}$ }! \\
 {\bfseries $\num{5678}$ } & \Verb!{\bfseries $\num{5678}$ }!\\
 \sisetup{
   detect-inline-weight = text,
   detect-all = true
 }
 {\boldmath  $ \num{5678}$ } & \Verb!{\boldmath $\num{5678}$ }! \\
 {\bfseries 你好 $\num{5678}$ } & \Verb!{\bfseries $\num{5678}$ }! 
   \end{tabu*}
\end{table}

% \cprotect\subsubsection{\sisetup}
\subsection{字体设置}

\subsection{解析数字}

\subsection{后处理数字}

\subsection{打印数字}


\end{document}
%%% Local Variables:
%%% mode: latex
%%% TeX-master: t
%%% End:
