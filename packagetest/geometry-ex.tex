\documentclass[a4paper]{article}
\usepackage[UTF8]{ctex}
\usepackage{lipsum}
\usepackage{kantlipsum}
\usepackage[ paperheight  =297mm,paperwidth   =210mm,  % or: "paper=a4paper"
             % layoutheight =197mm,layoutwidth  =130mm,
             % layoutvoffset= 50mm,layouthoffset= 40mm,
             margin=3cm, %includeheadfoot,
             % showframe=true,
             showcrop=true]{geometry}
\usepackage{pdflscape} % 注释掉这一行再编译一下,对比下效果
\usepackage{lscape}
\usepackage{everypage}

% 这部分改写\Lpagenumber,是为了在旋转页面时,同时将页码置于页面底部
% https://tex.stackexchange.com/questions/278113/single-landscape-page-with-page-number-at-the-bottom
\newcommand{\Lpagenumber}{\ifdim\textwidth=\linewidth\else\bgroup
  \dimendef\margin=0 %use \margin instead of \dimen0
  \ifodd\value{page}\margin=\oddsidemargin
  \else\margin=\evensidemargin
  \fi
  \raisebox{\dimexpr -\topmargin-\headheight-\headsep-0.5\linewidth}[0pt][0pt]{%
    \rlap{\hspace{\dimexpr \margin+\textheight+\footskip}%
    \llap{\rotatebox{90}{\thepage}}}}%
\egroup\fi}
\AddEverypageHook{\Lpagenumber}%

\begin{document}

宏包lipsum是从公元前45年的古典拉丁文学著作中截取150个段落中挑选段
落。\verb!\lipsum!命令用于显示这部著作中的若干段落,常用于测试排版效果。

类似的还有一个宏包\verb!kantlipsum!

\lipsum[1-3]

\newgeometry{left=3cm,right=2cm,bottom=3cm, twocolumn,landscape}
\savegeometry{mylandscape}

注意,虽然,在上个命令选项中使用了landscape,但页面并没有转为横向。
\lipsum[1-3]

\restoregeometry

现在恢复了原来的页面布置。

\lipsum[1-3]

\begin{landscape}
\pagestyle{empty}%
  要旋转页面,需要使用landscape包。这部分内容会自成一页,视角也会旋转过
  来。但问题是,页码仍然会随着页码旋转过去。
\end{landscape}
\pagestyle{plain}%

\restoregeometry

\kant[1-3]

\end{document}

%%% Local Variables:
%%% mode: latex
%%% TeX-master: t
%%% End:
